\usepackage{amsfonts}
\usepackage[none]{hyphenat} % PER NON FAR ANDARE A CAPO LE PAROLE CON IL TRATTINO
\usepackage[nottoc,notlot,notlof]{tocbibind} % boh
\usepackage{graphicx}
\usepackage{float}
\usepackage{centernot} % serve per il 'A NON implica B' nel comando \centernot
\usepackage{wrapfig}
\pdfsuppresswarningpagegroup=1

% FIGURE MATHCHA.IO
\usepackage{amsmath}
\usepackage{tikz}
\usepackage{mathdots}
\usepackage{yhmath}
\usepackage{cancel}
\usepackage{color}
\usepackage{siunitx}
\usepackage{array}
\usepackage{multirow}
\usepackage{amssymb}
\usepackage{gensymb}
\usepackage{tabularx}
\usepackage{booktabs}
\usepackage{caption}\captionsetup{belowskip=12pt,aboveskip=4pt}
\usetikzlibrary{fadings}
\usetikzlibrary{patterns}
\usetikzlibrary{shadows.blur}
\usepackage{placeins} % The placeins package gives the command \FloatBarrier, which will make sure any floats will be put in before this point
\usepackage{flafter}  % The flafter package ensures that floats don't appear until after they appear in the code.

% PER INCLUDERE FIGURE
\usepackage{import}
\usepackage{pdfpages}
\usepackage{transparent}
\usepackage{xcolor}

% QUESTO NUOVO COMANDO SERVE PER INCLDERE LE FIGURE FATTE IN INKSCAPE, CHE SI DEVONO TROVARE NELLA STESSA DIRECTORY DENTRO LA CARTELLA figures
\newcommand{\incfig}[2][1]{%
	\def\svgwidth{#1\columnwidth}
	\import{./figures/}{#2.pdf_tex}
}

% RISCRITTURA DI COMANDI
%\renewcommand{\epsilon}{\varepsilon} % NON USATO
%\renewcommand{\theta}{\vartheta} % NON USATO
%\renewcommand{\rho}{\varrho}
%\renewcommand{\phi}{\varphi} % NON USATO
\renewcommand{\degree}{^\circ\text{C}} % SIMBOLO GRADI
\newcommand{\notimplies}{\mathrel{{\ooalign{\hidewidth$\not\phantom{=}$\hidewidth\cr$\implies$}}}}

\usepackage{mathtools} % SERVE PER I DUE COMANDI DOPO
\DeclarePairedDelimiter{\abs}{\lvert}{\rvert} % CREA UN COMANDO abs()
\DeclarePairedDelimiter{\norma}{\lVert}{\rVert} % CREA UN COMANDO norma()

% INDICE
\setcounter{secnumdepth}{3} % DI DEFAULT LE SUBSUBSECTION NON SONO NUMERATE, COSÌ SÌ
\setcounter{tocdepth}{1} % FISSA LA PROFONDITÀ DELLE COSE MOSTRATE NELL'INDICE
%\usepackage{tocstyle}
%\usetocstyle{standard}
\usepackage[hidelinks]{hyperref} % RENDE L'INDICE INTERATTIVO E hidelinks NASCONDE IL BORDO ROSSO DAI RIFERIMENTI

% APPENDICI
\usepackage[toc,page]{appendix}
\newcommand{\nocontentsline}[3]{} % QUESTO COMANDO E QUELLO DOPO SERVONO PER AVERE IL COMANDO \tocless DA METTERE PRIMA DI UNA SEZIONE CHE NON VOGLIO FAR APPARIRE NELL'INDICE
\newcommand{\tocless}[2]{\bgroup\let\addcontentsline=\nocontentsline#1{#2}\egroup}

% FONT
\usepackage[T1]{fontenc}
\usepackage[utf8]{inputenc}
\usepackage[italian]{babel}
\usepackage{latexsym}
\usepackage{textcomp}
\usepackage{siunitx}

% SCRITTA LaTeX
\newcommand{\latex}{\LaTeX\xspace}
\newcommand{\tex}{\TeX\xspace}

% PARAGRAFI, INTERLINEA E MARGINE
\emergencystretch 3em % PER EVITARE CHE IL TESTO VADA OLTRE I MARGINI
\parindent 0ex % TOGLIE INTENDAMENTO PARAGRAFI
% \setlength{\parindent}{4em} % CAMBIA INDENTAMENTO PARAGRAFI
\setlength{\parskip}{\baselineskip} % CAMBIA SPAZIO TRA PARAGRAFI (POSSO METTERE ANCHE 1em)
% \renewcommand{\baselinestretch}{1.5} % CAMBIA INTERLINEA
% \usepackage[margin=1in]{geometry} % CAMBIA MARGINE DEL DOCUMENTO

% HEADER
\usepackage{fancyhdr}
\pagestyle{fancy}
\fancyhead{} % PULISCI HEADER
\fancyfoot{} % PULISCI FOOTER

% TOGLI I PROSISMI DUE COMMENTI PER CAMBIARE DA 'Capitolo X. Blabla' A 'X. Blabla'
%\renewcommand{\chaptermark}[1]{\markboth{#1}{}}
\fancyhead[RE]{\nouppercase{\leftmark}} %\fancyhead[RE]{\thechapter.  \leftmark}
\fancyhead[LO]{\nouppercase{\rightmark}}
\fancyhead[LE,RO]{\thepage}

%\fancyhead[L]{\slshape \MakeUppercase{FISICA SPERIMENTALE 2}}
%\fancyhead[R]{\slshape A cura di Teo Bucci e Giulia Di Giusto}
%\fancyfoot[C]{\thepage}
%\renewcommand{\headrulewidth}{0pt} % CANCELLA LINEA ORIZZONTALE HEAD

% CENTRARE VERTICALMENTE IL TITOLO DELLA PAGINA PRINCIPALE
\usepackage{titling}
\renewcommand\maketitlehooka{\null\mbox{}\vfill}
\renewcommand\maketitlehookd{\vfill\null}

% BOX DI VARIO TIPO

% TEOREMA TIPO 1
\usepackage{mdframed}
\newmdtheoremenv{mdtheo}{Teorema}

% BOX DEFINIZIONI/FORMULE IMPORANTI/TEOREMA TIPO 2
\usepackage{tcolorbox}
\usepackage{blindtext}
\usepackage{tikz,tkz-tab,amsmath}
\tcbuselibrary{theorems}

% FORMULE
%\newtcolorbox{formula}{
%	colback=red!5!white,
%	colframe=red!75!black
%	}
\newtcolorbox{formula}{
	colback=black!5!white,
	colframe=black
	}

% TEOREMA TIPO 2
\newtcbtheorem
	[number within=section]
	{theo}
	{Teorema}
	{
		colback=green!5,
		colframe=green!35!black,
		fonttitle=\bfseries,
		separator sign none % toglie i :
	}
	{th}

% DEFINIZIONI
\newtcbtheorem
	[number within=section] % init options
	{definition} % nome da scrivere in LaTeX
	{Definizione}% Titolo che verrà visualizzazione
	{
		colback=blue!5,
		colframe=blue!45!black,
		fonttitle=\bfseries,
	} % options
	{def} % PREFISSO/LABEL PER LE REF